\documentclass[12pt]{article}
\usepackage{graphicx}
\usepackage{color}
\usepackage{datetime}
\usepackage{nicefrac}


\topmargin      -0.0truein%do not change
\oddsidemargin   0.0truein%do not change
\evensidemargin  0.0truein%do not change
\textheight     8.5truein%do not change
\textwidth      6.3truein%do not change

\parskip 0.3cm
\parindent=0pt

%\pagestyle{empty}

\begin{document}
%%%%%%%%%%%%%%%%%%%%%%%%%%%%%%%%%%%%%

\vspace*{.5cm} \hrule

\def\sender{
  \it{Amit Aides}\\
  \scriptsize{Dept. of Electrical Engineering, Technion - Israel Inst.~Technology, Haifa 32000, Israel.}\\
  \scriptsize{Tel No.: (+972) 54-4526416}\hspace{5pt}\\
  \scriptsize{E-mail: amitibo@tx.technion.ac.il} } {\sender}
%%%%%%%%%%%%%%%%%%%%%%%%%%%%%%%%%%%%%

\vspace{0.5cm}
\noindent
To:\\
Dr. James Churnside,\\
Editor,\\
Associate Editor, Optics Express\\[0.6cm]

\begin{flushright}
  \today
\end{flushright}

\vspace{0.3cm} \noindent

Dear Dr. Churnside,

We are pleased to submit our revised manuscript {\em Multi sky-view 3D
  aerosol distribution recovery}, (195261).  We have addressed all the
requests made by the reviewers. The changes made to the manuscript are
detailed in the following pages. We thank the reviewers for their
helpful comments and hope that the paper in its current form satisfies
your requirements.

We look forward to hearing from you,\\

\vspace{0.2cm} Sincerely,

\vspace{0.6cm}
Amit Aides and Yoav Schechner\\

%%%%%%%%%%%%%%%%%%%%%%%%%%%%%%%%
% if the letter runs to a second page, or longer, the following lines
% must be inserted on the second page to prevent the heading and
% footer from appearing
\newpage

{\Large Changes Following the Comments by the Reviewers}\\

We would like to thank the reviewers for their general approval. In
the following we describe the changes made to the manuscript following
their remarks. Each reviewer's comment is summarized (colored in
blue), followed by our reply.

%%%%%%%%%%%%%%%%%%%%%%%%%%%%%%%%
% REVIEWER I
%%%%%%%%%%%%%%%%%%%%%%%%%%%%%%%%
\noindent \underline {\bf Reviewer 1}

We appreciate the reviewer's comment. We address it below.

\begin{enumerate}

\item \textcolor{blue}{ ``The paper can be devided in two parts, one
    for computing single scattering, the other for multiple
    scattering. While the first part can be treated by a simple
    inverse transformation sampling as presented by the authors, the
    second part should have a different more complex inversion. For
    this reason I think that larger errors arise in multiple
    scattering. I feel the authors should dedicate some time to review
    this second part, that is more realistic."}

  {}

\end{enumerate}

%%%%%%%%%%%%%%%%%%%%%%%%%%%%%%%%
% REVIEWER II
%%%%%%%%%%%%%%%%%%%%%%%%%%%%%%%%
\noindent \underline {\bf Reviewer 2}

Reviewer 2 sees the paper as novel, highly relevant and well written.
The reviewer suggests that the paper is well worth publishing and
recommends accepting the paper after a minor revision.  We thank the
reviewer for his general approval. Below we address his/her comments:

\begin{enumerate}

%%%%%%%%%%%%%%%%
\item \textcolor{blue}{``I don’t understand Eq 9. If $F(u) =
    1-\exp(-u)$ how is $\nicefrac{1}{F} = -\ln(1-u)$?''}

  {There is no contradiction. The notation $F^{-1}$ refers to the
    inverse function of function $F$ and not to $\nicefrac{1}{F}$}

%%%%%%%%%%%%%%%%
\item \textcolor{blue}{ ``Maybe define $\mu$ and voxel''}

  {Following the reviewer's comment, we have defined $\mu$ after
    Eq.~5, and revised the definition of the term ``voxel''. }

%%%%%%%%%%%%%%%%

\item \textcolor{blue}{ ``How do you predefine the voxels
    vertically?''}

  {}

%%%%%%%%%%%%%%%%

\item \textcolor{blue}{ ``The result of the example (sec. 8) could be
    a little more precise in the sense that non-specialists get a
    clear idea of how your method works and which aerosol properties
    hinder or favor retrieval.''}

  {}

%%%%%%%%%%%%%%%%

\item \textcolor{blue}{ ``How many cameras in which distance do you
    need to obtain feasible results for the case of Sec. 5?''}

  {}

\end{enumerate}

\end{document}
