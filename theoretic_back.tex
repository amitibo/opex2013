\documentclass[article.tex]{subfiles} 
\begin{document}

\section{Theoretical Background}
\label{sec:back}

\noindent {\bf Extinction}: Sun rays (SR) irradiate a small volume
that includes particles of a certain type.  Each particle has an {\em
  extinction cross section} $\sigma$ (units \si{\meter\squared}) for
interacting with the irradiance. The volume includes many particles of
this type. The denser the volume, the stronger is the extinction.  The
density of the particles is $n$ (units
\si[sticky-per]{\per\cubic\metre}). Per unit volume, the {\em
  extinction coefficient} $\beta$ (units \si{\per\meter}) is
\begin{align}
  \beta= \sigma n \;.
  \label{eq:extinctc}
\end{align}
The volume has infinitesimal length $dl$. Then, the relative portion
of extinct SR irradiance is the unitless differential {\em optical
  depth}
\begin{align}
  d\tau= \beta dl=\sigma n dl \;.
  \label{eq:extinct}
\end{align}
The optical depth aggregates in extended propagation:
\begin{align}
  \tau= \int d\tau=\int \beta dl=\int \sigma n dl \;.
  \label{eq:tau}
\end{align}
Through an attenuating volume, the {\em transmittance} exponentially
decays with the optical depth:
\begin{align}
  t=\exp^{-\tau}
  \label{eq:beer-lambert}
\end{align}

\noindent {\bf Scattering}: Interaction of a single particle with the
irradiance is by absorption and scattering. The weight of scattering
(to all directions), relative to the total extinction is given by the
unitless {\em single scattering albedo} $\varpi$ of the particle. The
{\em scattering coefficient} $\alpha$ (units \si{\per\meter}) is
\begin{align}
  \alpha= \varpi\beta=\varpi \sigma n \;.
  \label{eq:alph}
\end{align}
Scattering is generally anisotropic. Its angular distribution is
determined by a {\em phase function} $P$. Part of the light scatters
towards a camera's line of sight (LOS), as illustrated in
\cref{fig:groundgrid}. The angle between the SR and LOS is the
scattering angle $\Phi^{\rm scatter}$. The phase function $P(\Phi^{\rm
  scatter})$ is normalized: its integral over all solid angles is
unit. The {\em angular scattering coefficient} $\tilde\alpha$ (units
\si[sticky-per]{\per\meter\steradian}) is
\begin{align}
  \tilde\alpha(\Phi^{\rm scatter}) = \varpi\beta P(\Phi^{\rm scatter})
  = \varpi \sigma n P(\Phi^{\rm scatter}).
  \label{eq:alphabasic}
\end{align}

The phase function is often approximated by a parametric expression,
as the Henyey-Greenstein~\cite{Cornette1995} function\todo{Correct the Henyey-Greenstein function},
\begin{align}
  P_g(\Phi^{\rm scatter})\approx \frac{3} {8\pi} \frac{(1 -
    g^2)(1+\mu^2)} {(2+g^2)(1 + g^2 - 2g\mu)^\frac{3}{2}}
  \label{eq:aerosol_scatter}
\end{align}
where $g$ is an {\em anisotropy parameter} and
\begin{align}
  \mu\equiv \cos \Phi^{\rm scatter}.
  \label{eq:mu}
\end{align}

\noindent {\bf Air molecules}: Scattering by air molecules follows the
{\em Rayleigh} model:
\begin{align}
  P^{\rm air}\left[\mu(k)\right] &\approx \frac{3}{16\pi}(1+\mu^2)
  \label{eq:rayleighP}
\end{align}
and $\varpi^{\rm air}$=1. Air molecular density $n^{\rm air}$ falls
off approximately exponentially with altitude, with a
characteristic~\cite{Levi1980} falloff height $H^\mathrm{air}=8\
\si{\km}$. Consequently~\cite{Levi1980}, the coefficients for
extinction and scattering by air molecules are modeled by
\begin{align}
  \alpha^{\rm air}(h, \lambda)=\beta^{\rm air}(h, \lambda) \approxeq
  \frac{1.09 \times 10^{-3}}{\lambda^4}
  \exp(-h/H^\mathrm{air}) %\left[-\frac{h}{H^\mathrm{air}}\right].
  \label{eq:rayleighbeta}
\end{align}


\end{document}
% -----------------------------------------------------------
