\documentclass[12pt]{article}
\usepackage{graphicx}
\usepackage{color}
\usepackage{datetime}
\usdate
\usepackage{nicefrac}

%% fonts
\usepackage{mathptmx,courier,textcomp}
\usepackage{amsmath}
\usepackage{amssymb}
\usepackage{bm}
\usepackage{xfrac}


\DeclareMathAlphabet\mathbfcal{OMS}{cmsy}{b}{n}


%
% Some new commands I use in this text
%
% Note:
% Nice curly font is {\bm{\mathcal{D}}}
%
\newcommand{\Grad}[1]{\bm{\triangledown_{#1}}}
\newcommand{\abbrev}[1]{\rm{#1}}
\newcommand{\argmin}{\mathrm{arg}\min}
\newcommand{\curly}[1]{\left\{#1\right\}}
\newcommand{\roundy}[1]{\left(#1\right)}
\newcommand{\recty}[1]{\left[#1\right]}
\newcommand{\PartDeriv}[2]{\frac{\partial{#1}}{\partial{#2}}}
\newcommand{\vect}[1]{\bm{#1}}
\newcommand{\mat}[1]{\bm{#1}}
\newcommand{\transpose}[1]{{#1}^\intercal}
\newcommand{\derivsym}[1]{\,d{#1}}
\newcommand{\yoavcomment}[1]{}
\renewcommand{\yoavcomment}[1]{#1} % Comment to remove images
\newcommand{\fix}[2]{\st{#1}\hl{#2}}

%
% Used symbols (only partial for now...)
%
\newcommand{\OpSphere}{\mathbfcal{S}}
\newcommand{\OpRot}{\mathbfcal{R}}
\newcommand{\OpDistance}{\bm{D}}
\newcommand{\OpCumsum}{\mathbfcal{C}}
\newcommand{\OpInt}{\mathbfcal{I}}
\newcommand{\OpCamera}{\mathbfcal{P}}
\newcommand{\MaskSun}{\mathbfcal{M}}
\newcommand{\Laplacian}{\mathbfcal{L}}
\newcommand{\OpDiag}[1]{\mathbb{D}\left\{#1\right\}}
\newcommand{\DistSet}{\mathcal{C}}
\newcommand{\DistUnknown}{\vect{n}}
\newcommand{\DistEstimated}{\hat{\vect{n}}}
\newcommand{\CostFunc}[1]{E(#1)}


\topmargin      -0.0truein%do not change
\oddsidemargin   0.0truein%do not change
\evensidemargin  0.0truein%do not change
\textheight     8.5truein%do not change
\textwidth      6.3truein%do not change

\parskip 0.3cm
\parindent=0pt

%\pagestyle{empty}

\begin{document}
%%%%%%%%%%%%%%%%%%%%%%%%%%%%%%%%%%%%%

\vspace*{.5cm} \hrule

\def\sender{
  \it{Amit Aides}\\
  \scriptsize{Dept. of Electrical Engineering, Technion - Israel Inst.~Technology, Haifa 32000, Israel.}\\
  \scriptsize{Tel No.: (+972) 54-4526416}\hspace{5pt}\\
  \scriptsize{E-mail: amitibo@tx.technion.ac.il} } {\sender}
%%%%%%%%%%%%%%%%%%%%%%%%%%%%%%%%%%%%%

\vspace{0.5cm}
\noindent
To:\\
Dr. James Churnside,\\
Associate Editor, Optics Express\\[0.6cm]

\begin{flushright}
  \today
\end{flushright}

\vspace{0.3cm} \noindent

Dear Dr. Churnside,

We are pleased to submit our revised manuscript {\em Multi sky-view 3D
  aerosol distribution recovery}, (195261).  We have addressed all the
requests made by the reviewers. The changes made to the manuscript are
detailed in the following pages. We thank the reviewers for their
helpful comments and hope that the paper in its current form satisfies
your requirements.

We look forward to hearing from you,\\

\vspace{0.2cm} Sincerely,

\vspace{0.6cm}
Amit Aides, corresponding author.\\

%%%%%%%%%%%%%%%%%%%%%%%%%%%%%%%%
% if the letter runs to a second page, or longer, the following lines
% must be inserted on the second page to prevent the heading and
% footer from appearing
\newpage

{\Large Changes Following the Comments by the Reviewers}\\

We would like to thank the reviewers for their general approval and useful
suggestions. In the following we describe the changes made to the manuscript
following their remarks. Each comment is summarized (colored in
blue), followed by our reply.

%%%%%%%%%%%%%%%%%%%%%%%%%%%%%%%%
% REVIEWER I
%%%%%%%%%%%%%%%%%%%%%%%%%%%%%%%%
\noindent \underline {\bf Reviewer 1}

We appreciate the reviewer's comment. We address it below.

\begin{enumerate}

\item \textcolor{blue}{ ``The paper can be divided in two parts, one
    for computing single scattering, the other for multiple
    scattering. While the first part can be treated by a simple
    inverse transformation sampling as presented by the authors, the
    second part should have a different more complex inversion. For
    this reason I think that larger errors arise in multiple
    scattering. I feel the authors should dedicate some time to review
    this second part, that is more realistic."}

  {
    In this article we describe a novel data acquisition system
    and suggest a method to recover the 3D atmosphere based on
    the single-scattering approximation. Following the reviewer's
    suggestion we expanded our reference to the implications
    of high order scattering on our method.

    We added the new Fig.~5. In this figure we visually demonstrate
    the extents to which the single-scattering approximation is valid
    and to which it is not.

    The newly added Fig.~6 shows the contribution of each scattering order to
    the final MC image. Figure~6 makes clear that there are regions in
    the image that are affected by multiple scattering more than others.
    The two new paragraphs at the end of Sec.~6 reference the issues
    raised by Fig.~5 and Fig.~6.

    In the top of p.~10 we added a new description of $MaskSun$. We use
    $MaskSun$ to mask the regions in the image which are dominated 
    by multi-scattering as shown by Fig.~6. This method improves the
    results of the reconstruction algorithm.

    We test our reconstruction algorithm on a new scene where we
    increase the aerosol density tenfold. This significantly increases the
    effects of high order scattering. The results are shown in 
    Figs.~9(d) and 10(c), and reported in Table~1.
    Paragraph ``Density scale'' in Sec.~8 summarizes this experiment.
    Further discussion is given in Sec.~9.
  }

\end{enumerate}

\newpage

%%%%%%%%%%%%%%%%%%%%%%%%%%%%%%%%
% REVIEWER II
%%%%%%%%%%%%%%%%%%%%%%%%%%%%%%%%
\noindent \underline {\bf Reviewer 2}

Reviewer 2 sees the paper as novel, highly relevant and well written.
The reviewer suggests that the paper is well worth publishing and
recommends accepting the paper after a minor revision.  We thank the
reviewer for his general approval. Below we address his/her comments:

\begin{enumerate}

%%%%%%%%%%%%%%%%
\item \textcolor{blue}{``I don’t understand Eq 9. If $F(u) =
    1-\exp(-u)$ how is $\nicefrac{1}{F} = -\ln(1-u)$?''}

  {The notation $F^{-1}$ refers to the inverse function
   of function $F$ and not to $\nicefrac{1}{F}$. Following the
   reviewer's comment we added a clarification in the ``Inverse
   transform sampling'' paragraph on p.~4. }

%%%%%%%%%%%%%%%%
\item \textcolor{blue}{ ``Maybe define $\mu$ and voxel''}

  {We thank the reviewer for turning our attention to this.
   We replaced $\mu$ in Eq.~5 with its definition,
   $\cos \Phi^{\rm scatter}$. We revised the definition
   of the term ``voxel'', at the beginning of Sec.~4 of the
   article.}

%%%%%%%%%%%%%%%%

\item \textcolor{blue}{ ``How do you predefine the voxels
    vertically?''}

  {
    We now explain our vertical division of the atmosphere at the
    beginning of Sec.~5.
  }

%%%%%%%%%%%%%%%%

\item \textcolor{blue}{ ``The result of the example (sec. 8) could be
    a little more precise in the sense that non-specialists get a
    clear idea of how your method works and which aerosol properties
    hinder or favor retrieval.''}

  {
    We appreciate the reviewer's comment. We added new simulations to
    test the effect of varying different properties of the aerosols.
    These experiments are detailed in Sec.~8. The results are shown
    in Figs.~9 and 10 and reported in Table.~1.
  }

%%%%%%%%%%%%%%%%

\item \textcolor{blue}{ ``How many cameras in which distance do you
    need to obtain feasible results for the case of Sec. 5?''}

  {Following this comment, we repeated the reconstruction
   simulations many times, each time varying the distance between
   cameras. These tests are described in the paragraph ``Density of viewpoints''
   in Sec.~8. The results are shown in Fig.~8.}

\end{enumerate}

\end{document}
